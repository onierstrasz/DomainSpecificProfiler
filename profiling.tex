% % % % % % % % % % % % % % % % % % % % % % % % % % % % % % % % % %
\documentclass[runningheads]{llncs}

% packages
\usepackage{xspace}
\usepackage{ifthen}
\usepackage{amsbsy}
\usepackage{amssymb}
\usepackage{balance}
\usepackage{booktabs}
\usepackage{graphicx}
\usepackage{multirow}
\usepackage{needspace}
\usepackage{microtype}
\usepackage{bold-extra}

% constants
\newcommand{\Title}{Domain-Specific Profiling}
\newcommand{\TitleShort}{\Title}
\newcommand{\Authors}{Alexandre Bergel, Lukas Renggli}
\newcommand{\AuthorsShort}{A. Bergel, L. Renggli}

% references
\usepackage[colorlinks]{hyperref}
\usepackage[all]{hypcap}
\setcounter{tocdepth}{2}
\hypersetup{
	colorlinks=true,
	urlcolor=black,
	linkcolor=black,
	citecolor=black,
	plainpages=false,
	bookmarksopen=true,
	pdfauthor={\Authors},
	pdftitle={\Title}}

\def\chapterautorefname{Chapter}
\def\appendixautorefname{Appendix}
\def\sectionautorefname{Section}
\def\subsectionautorefname{Section}
\def\figureautorefname{Figure}
\def\tableautorefname{Table}
\def\listingautorefname{Listing}

% source code
\usepackage{xcolor}
\usepackage{textcomp}
\usepackage{listings}
\definecolor{source}{gray}{0.9}
\lstset{
	language={},
	% characters
	tabsize=3,
	upquote=true,
	escapechar={!},
	keepspaces=true,
	breaklines=true,
	alsoletter={\#:},
	breakautoindent=true,
	columns=fullflexible,
	showstringspaces=false,
	basicstyle=\footnotesize\ttfamily,
	% background
	frame=single,
    framerule=0pt,
	backgroundcolor=\color{source},
	% numbering
	numbersep=5pt,
	numberstyle=\tiny,
	numberfirstline=true,
	% captioning
	captionpos=b,
	% formatting (html)
	moredelim=[is][\textbf]{<b>}{</b>},
	moredelim=[is][\textit]{<i>}{</i>},
	moredelim=[is][\color{red}\uwave]{<u>}{</u>},
	moredelim=[is][\color{red}\sout]{<del>}{</del>},
	moredelim=[is][\color{blue}\underline]{<ins>}{</ins>}}
\newcommand{\ct}{\lstinline[backgroundcolor=\color{white},basicstyle=\footnotesize\ttfamily]}
\newcommand{\lct}[1]{{\small\tt #1}}

% tikz
% \usepackage{tikz}
% \usetikzlibrary{matrix}
% \usetikzlibrary{arrows}
% \usetikzlibrary{external}
% \usetikzlibrary{positioning}
% \usetikzlibrary{shapes.multipart}
% 
% \tikzset{
% 	every picture/.style={semithick},
% 	every text node part/.style={align=center}}

% proof-reading
\usepackage{xcolor}
\usepackage[normalem]{ulem}
\newcommand{\ra}{$\rightarrow$}
\newcommand{\ugh}[1]{\textcolor{red}{\uwave{#1}}} % please rephrase
\newcommand{\ins}[1]{\textcolor{blue}{\uline{#1}}} % please insert
\newcommand{\del}[1]{\textcolor{red}{\sout{#1}}} % please delete
\newcommand{\chg}[2]{\textcolor{red}{\sout{#1}}{\ra}\textcolor{blue}{\uline{#2}}} % please change
\newcommand{\chk}[1]{\textcolor{ForestGreen}{#1}} % changed, please check

% comments \nb{label}{color}{text}
\newboolean{showcomments}
\setboolean{showcomments}{true}
\ifthenelse{\boolean{showcomments}}
	{\newcommand{\nb}[3]{
		{\colorbox{#2}{\bfseries\sffamily\scriptsize\textcolor{white}{#1}}}
		{\textcolor{#2}{\sf\small$\blacktriangleright$\textit{#3}$\blacktriangleleft$}}}
	 \newcommand{\version}{\emph{\scriptsize$-$Id$-$}}}
	{\newcommand{\nb}[2]{}
	 \newcommand{\version}{}}
\newcommand{\rev}[2]{\nb{Reviewer #1}{red}{#2}}
\newcommand{\ab}[1]{\nb{Alexandre}{blue}{#1}}
\newcommand{\lr}[1]{\nb{Lukas}{orange}{#1}}

% graphics: \fig{position}{percentage-width}{filename}{caption}
\DeclareGraphicsExtensions{.png,.jpg,.pdf,.eps,.gif}
\graphicspath{{figures/}}
\newcommand{\fig}[4]{
	\begin{figure}[#1]
		\centering
		\includegraphics[width=#2\textwidth]{#3}
		\caption{\label{fig:#3}#4}
	\end{figure}}

% abbreviations
\newcommand{\ie}{\emph{i.e.,}\xspace}
\newcommand{\eg}{\emph{e.g.,}\xspace}
\newcommand{\etc}{\emph{etc.}\xspace}
\newcommand{\etal}{\emph{et al.}\xspace}

% lists
\newenvironment{bullets}[0]
	{\begin{itemize}}
	{\end{itemize}}

\newcommand{\seclabel}[1]{\label{sec:#1}}

% D O C U M E N T
% % % % % % % % % % % % % % % % % % % % % % % % % % % % % % % % % %
\begin{document}

% T I T L E
% % % % % % % % % % % % % % % % % % % % % % % % % % % % % % % % % %

\title{\Title}
\titlerunning{\TitleShort}

\author{\Authors} 
\authorrunning{\AuthorsShort}

\institute{PLEIAD Lab, Department of Computer Science (DCC),\\
	University of Chile, Santiago, Chile}

\maketitle

% A B S T R A C T
% % % % % % % % % % % % % % % % % % % % % % % % % % % % % % % % % %

\begin{abstract}
	Domain-specific languages and models are increasingly used within general-purpose host languages. While traditional profiling tools perform well on host language code itself, they often fail to provide meaningful results if the developers start to abstract from traditional source code. In this paper we demonstrate the need of dedicated profiling tools with three different case-studies. Furthermore, we present an infrastructure that enables developers to quickly prototype new profilers for their domain-specific models or languages.
\end{abstract}

% % % % % % % % % % % % % % % % % % % % % % % % % % % % % % % % % %
\section{Introduction}\seclabel{introduction}

- Specific domain are very present in today software development

- Whereas research on domain-specific languages made great progresses on language specification, implementation and verification, little has been done on profiling.

- Comparison with previous work on domain-specific code checking \cite{Reng10b}.

- Our approach to profiling \cite{Berg10c}. Our strategy to apply it to specific domains (maybe with figure).

The contributions of this paper are: (1) the identification of the need for domain-specific profilers, (2) the presentation of various real-world case-studies where domain-specific profilers helped to significantly improve performance and correctness of domain-specific code, and (3) the presentation of an infrastructure for prototyping domain-specific profilers.

The remainder of this paper is structured as follows: \autoref{sec:problem} illustrates the problems of using a general-purpose profiler on code building on top of domain-specific language. \autoref{sec:profiler} introduces our approach to domain-specific profiling with three case-studies. \autoref{sec:implementation} presents our infrastructure to implement domain-specific profilers; and \autoref{sec:conclusion} summarizes the paper and discusses future work.

% % % % % % % % % % % % % % % % % % % % % % % % % % % % % % % % % %
\section{Problem}\seclabel{problem}

\subsection{Difficulty of profiling a specific domain}

- Example with MessageTally

- PetitParser coverage problem

\subsection{Summary of the problem}

Current profilers are useful to gather runtime data (\eg method invocations, method coverage, call trees, memory consumption, \etc) from the static code model offered by the programming language (\eg packages, classes, methods, statements, \etc). This works well for to profile concerns at the abstraction level of the programming language like execution time, invocation count, memory consumption, code coverage, flow analysis, \etc.

In many today's software there is a significant gap between the model offered by the execution platform and the model of the actually running application. The proliferation of meta-models and domain-specific languages builds new abstractions that map to the underlying execution platform in non-trivial ways. Traditional profiling tools fail to display relevant information in the presence of such abstractions.

% % % % % % % % % % % % % % % % % % % % % % % % % % % % % % % % % %

\section{Domain Specific Profiler}\seclabel{profiler}

%:========
\subsection{Domain Specific Profiler in a Nutshell}

\paragraph{}
 having the profiler aware of elements to be profiled, and not about their implementation

\paragraph{}
perform profiling on the abstraction of the domain, not in the host language

\paragraph{}
 assuming that each operation of the domain is constant in time


\begin{tabular}{c|c|c}
					&	\textbf{Host language} 	& \textbf{DSL} \\ \hline
					&						& 		\\
\textbf{$M_0$ - Static}	&	classes, methods		& 		\\
					&						& 		\\\hline
					&						& Mondrian\\
\textbf{$M_1$ - Dynamic}	&	method activation		& OB-metamodel\\
					&						& Grammars\\ \hline
					&						& OB Browser\\
\textbf{$M_2$ -} 		&						& Visualization\\
					&						& Mondrian
\end{tabular}

%:========
\subsection{Case Study: Caches in Mondrian}

\paragraph{Keeping track of the caches}

\begin{lstlisting}
execute
	| view allNodes |
	view := MOViewRenderer title: 'Debugging'.
	view interaction every: 1500 do: [ 
			view root clear.
			view shape rectangle
							if: #hasCachedForm fillColor: Color red;
							disableCache.
			allNodes := easel canvas root allNodes.
			view nodes: allNodes forEach: [:each |
				view nodes: each metricCaches.
				view gridLayout gapSize: 2.
			].  
			view edges: allNodes from: #owner to: #yourself.

			view treeLayout.
					
			view root applyLayout. ].	
	view open.
\end{lstlisting}

%:========
\subsection{Case Study: Events in Omnibrowser}

\paragraph{Meta-model instantiation}
\begin{lstlisting}
	model := OBSystemBrowser onClass: Object selector: #printOn:.
	profile := OmniProfiler 
		profile: [ window :=  model open ]
		inPackage: 'OmniBrowser'.
\end{lstlisting}

\paragraph{Recording events}
\begin{lstlisting}
	model := OBSystemBrowser onClass: Object.
	actualCounter := Dictionary new.
	previousCounter := Dictionary new.
	model announcer
		observe: OBAnnouncement
		do: [ :ann | actualCounter at: ann class put: (actualCounter at: ann class ifAbsentPut: [ 0 ]) + 1 ].
\end{lstlisting}


%:========
\subsection{Case Study: Parsing Framework with PetitParser}

\paragraph{Grammar coverage}
\begin{lstlisting}
	profile := PetitProfiler profile: [ PPSmalltalkGrammarTests suite run: TestResult new ] inPackage: 'PetitParser'.
	parser := PPParserResource current parserAt: PPSmalltalkGrammar.
	
	view := MOViewRenderer new.
	view shape rectangle 
		size: [:p | ((profile numberOfActivationsFor: p) + 1) log * 10 ];
		if: [:p | (profile numberOfActivationsFor: p) = 0] borderColor: Color red.
	view nodes: parser allParsers.
	view shape arrowedLine.
	view edgesToAll: #children.
	view treeLayout.
	view open
\end{lstlisting}

Check of the parseOn: method. 



% % % % % % % % % % % % % % % % % % % % % % % % % % % % % % % % % %
\section{Implementation}\seclabel{implementation}

Implemented with the Spy profiling framework

%:========
\subsection{Benchmark}

% % % % % % % % % % % % % % % % % % % % % % % % % % % % % % % % % %
\section{Conclusion}\seclabel{conclusion}

This problem is rather well generalized. For example, profiling a virtual machine is a particularly difficult activity, since instead of providing information about bytecode and primitive executions, the virtual machine talks about functions executions\footnote{\url{http://www.mirandabanda.org/cogblog/2008/12/30/the-idee-fixe-and-the-perfected-profiler/}}.

% % % % % % % % % % % % % % % % % % % % % % % % % % % % % % % % % %
\section*{Acknowledgments}

\small We gratefully thanks ...

% bibliography
% % % % % % % % % % % % % % % % % % % % % % % % % % % % % % % % %
\bibliographystyle{splncs}
\bibliography{scg}

\end{document}